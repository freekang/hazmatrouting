%% This is file `elsarticle-template-1a-num.tex',
%%
%% Copyright 2009 Elsevier Ltd
%%
%% This file is part of the 'Elsarticle Bundle'.
%% ---------------------------------------------
%%
%% It may be distributed under the conditions of the LaTeX Project Public
%% License, either version 1.2 of this license or (at your option) any
%% later version.  The latest version of this license is in
%%    http://www.latex-project.org/lppl.txt
%% and version 1.2 or later is part of all distributions of LaTeX
%% version 1999/12/01 or later.
%%
%% The list of all files belonging to the 'Elsarticle Bundle' is
%% given in the file `manifest.txt'.
%%
%% Template article for Elsevier's document class `elsarticle'
%% with numbered style bibliographic references
%%
%% $Id: elsarticle-template-1a-num.tex 151 2009-10-08 05:18:25Z rishi $
%% $URL: http://lenova.river-valley.com/svn/elsbst/trunk/elsarticle-template-1a-num.tex $
%%
\documentclass[preprint,12pt]{elsarticle}

%% Use the option review to obtain double line spacing
%% \documentclass[preprint,review,12pt]{elsarticle}

%% Use the options 1p,twocolumn; 3p; 3p,twocolumn; 5p; or 5p,twocolumn
%% for a journal layout:
%% \documentclass[final,1p,times]{elsarticle}
%% \documentclass[final,1p,times,twocolumn]{elsarticle}
%% \documentclass[final,3p,times]{elsarticle}
%% \documentclass[final,3p,times,twocolumn]{elsarticle}
%% \documentclass[final,5p,times]{elsarticle}
%% \documentclass[final,5p,times,twocolumn]{elsarticle}

%% if you use PostScript figures in your article
%% use the graphics package for simple commands
%% \usepackage{graphics}
%% or use the graphicx package for more complicated commands
%% \usepackage{graphicx}
%% or use the epsfig package if you prefer to use the old commands
%% \usepackage{epsfig}

%% The amssymb package provides various useful mathematical symbols
\usepackage{amssymb}
%% The amsthm package provides extended theorem environments
%% \usepackage{amsthm}

%% The lineno packages adds line numbers. Start line numbering with
%% \begin{linenumbers}, end it with \end{linenumbers}. Or switch it on
%% for the whole article with \linenumbers after \end{frontmatter}.
%% \usepackage{lineno}

%% natbib.sty is loaded by default. However, natbib options can be
%% provided with \biboptions{...} command. Following options are
%% valid:

%%   round  -  round parentheses are used (default)
%%   square -  square brackets are used   [option]
%%   curly  -  curly braces are used      {option}
%%   angle  -  angle brackets are used    <option>
%%   semicolon  -  multiple citations separated by semi-colon
%%   colon  - same as semicolon, an earlier confusion
%%   comma  -  separated by comma
%%   numbers-  selects numerical citations
%%   super  -  numerical citations as superscripts
%%   sort   -  sorts multiple citations according to order in ref. list
%%   sort&compress   -  like sort, but also compresses numerical citations
%%   compress - compresses without sorting
%%
%% \biboptions{comma,round}

% \biboptions{}

%%%%%%%%%%%%%%%%%%%%%%%%%%%%%%%%%
% our packages and definitions: %
%%%%%%%%%%%%%%%%%%%%%%%%%%%%%%%%%
\usepackage{url} 
\usepackage{color}
% \usepackage{dsfont}
\usepackage{booktabs}
\usepackage{amsmath}
\usepackage{multirow}

\newcommand{\TODO}[1]{{\color{red}TODO: #1}}
\newcommand{\TODON}[1]{{\color{blue}TODO: #1}}
\newcommand{\COMMENTD}[1]{{\color{green}Dimo: #1}}
\newcommand{\COMMENTN}[1]{{\color{green}Nora: #1}}

\renewcommand{\topfraction}{.70}
\renewcommand{\textfraction}{.20}

\newcommand{\reviewer}[1]{\begin{flushleft}\noindent{\sloppy {\bf Comment: } #1}\end{flushleft}}
\newcommand{\answer}[1]{\noindent{\em {\bf Reply: } #1}\\[1em]}

\journal{DAM}

\begin{document}

\begin{frontmatter}

%% Title, authors and addresses

%% use the tnoteref command within \title for footnotes;
%% use the tnotetext command for the associated footnote;
%% use the fnref command within \author or \address for footnotes;
%% use the fntext command for the associated footnote;
%% use the corref command within \author for corresponding author footnotes;
%% use the cortext command for the associated footnote;
%% use the ead command for the email address,
%% and the form \ead[url] for the home page:
%%
%% \title{Title\tnoteref{label1}}
%% \tnotetext[label1]{}
%% \author{Name\corref{cor1}\fnref{label2}}
%% \ead{email address}
%% \ead[url]{home page}
%% \fntext[label2]{}
%% \cortext[cor1]{}
%% \address{Address\fnref{label3}}
%% \fntext[label3]{}

\title{An Evolutionary Algorithm for the Multiobjective Risk-Equity Constrained Routing Problem\\ Responses to the Reviewers' Comments}

%% use optional labels to link authors explicitly to addresses:
%% \author[label1,label2]{<author name>}
%% \address[label1]{<address>}
%% \address[label2]{<address>}

\author{Dimo Brockhoff}

\author{Nora Touati-Moungla}



\begin{abstract}
tbd: thank reviewers for careful comments, etc. 
\end{abstract}

\end{frontmatter}


%%%%%%%%%%%%%%%%%%%%%%%%%%%%%%%%%%%%%%%%%%%%%%%%%%%%%%%%%%%%%%%%%%%%%%%%%
\section*{Reviewer 1}
%%%%%%%%%%%%%%%%%%%%%%%%%%%%%%%%%%%%%%%%%%%%%%%%%%%%%%%%%%%%%%%%%%%%%%%%%
\reviewer{
It is not very clear what the authors mean for ``minimization of the equity risk'' (see, e.g., abstract,
and Introduction (three times) ) or ``minimizing the distributed risk'' (see, e.g., Section 2.1). As far I
can understand, one wishes to maximize risk equity.
}
\answer{\TODO{tbd}}

\reviewer{
It is not clear why the authors talk about ``risk-equity constrained routing problem'' (see, e.g.,
Title, Introduction, Sections 2, 3, 3.5, and 4). In their multi-objective model, risk equity does not
appear as a constraint, but as one of the objectives of the problem.
}
\answer{\TODO{tbd}}

\reviewer{
There is a lack in reviewing existing approaches:\\
a) Besides the cited paper of Dell'Olmo et al. (2005), more recently other papers propose
approaches for finding Pareto-optimal dissimilar paths for hazmat routing. See, e.g.:

M. Caramia and S. Giordani. On the selection of k efficient paths by clustering techniques.
International Journal on Data Mining, Modelling and Management, 1(3):237.260, 2009.\\
M. Caramia, S. Giordani, and A. Iovanella. On the selection of k routes in multi-objective
hazmat route planning. IMA Journal of Management Mathematics, 21(3):239.251, 2010.

b) The model studied by the authors is multi-objective, but it considers hazmat routing from the
perspective of a single decision maker. Indeed global hazmat route planning involves in general
at least two stakeholders: the government authority from one side and the carriers from the other
side. The main concern for the government authority is to control the risk induced by hazmat
transportation over the population and the environment, while the main concern from the
carriers�f point of view is minimizing transportation cost. Since, typically, the government
authority does not have the right to impose specific routes to hazmat carriers (contrarily to the
assumption made by single decision maker hazmat routing models like the one proposed by the
authors), it can only achieve hazmat transportation risk mitigation adopting policies that induce
hazmat carriers to select low risk routes. This has led researchers to study multi-decision-makers
models for hazmat routing. In particular, two classes of models are investigated in recent papers:
``hazmat transportation network design'''' and ``toll setting approaches''. The authors should
therefore recall also these classes of models for global hazmat route planning and cite the related
papers, e.g.:

B.Y. Kara and V. Verter. Designing a road network for hazardous materials transportation.
Transportation Science, 38(2):188--196, 2004.\\
E. Erkut and O. Alp. Designing a road network for dangerous goods shipments. Computers \&
Operations Research, 34(5):1389--1405, 2007.\\
E. Erkut and F. Gzara. Solving the hazmat transport network design problem. Computers \&
Operations Research, 35(7):2234--2247, 2008.\\
V. Verter and B.Y. Kara. A path-based approach for the hazardous network design problem.
Management Science, 54(1):29--40, 2008.\\
L. Bianco, M. Caramia, and S. Giordani. A bilevel flow model for hazmat transportation
network design. Transportation Research Part C, 17:175--196, 2009.\\
P. Marcotte, A. Mercier, G. Savard, and V. Verter. Toll policies for mitigating hazardous
materials transport risk. Transportation Science, 43(2): 228--243, 2009.
}
\answer{\TODO{tbd}}

\reviewer{
Model's parameter $r^c_{ij}$ is not defined.
}
\answer{\TODO{tbd}}

\reviewer{
Please clarify that with o.f. (3) one achieves the maximization of the distribution of the risk, i.e.
risk equity.
}
\answer{\TODO{tbd}}

\reviewer{
Note that (3) can be linearized by introducing a new variable (e.g., variable $\gamma$), the constraints $\sum_{c\in C} r^c_{ij}y^c_{ij} \leq \gamma$, for all $(i, j)\in A$, and by replacing (3) with min $\gamma$.
}
\answer{\TODO{tbd}}

\reviewer{
In Section 3.1, some details explaining the meaning of ``variable length representation'' should be
added.
}
\answer{\TODO{tbd}}

\reviewer{
In Section 3.3, it is not clear the meaning of ``length(p)''. As far I can understand it should be
intended as the number of vertices of path p.
}
\answer{\TODO{tbd}}

\reviewer{
The authors provide sufficient insights about initialization and variation of the solutions'
population, but not how the population is updated and the stopping criteria used. Please provide a
pseudo-code description of the whole (HypE) algorithm, in order to make the paper self-contained.
}
\answer{\TODO{tbd}}

\reviewer{
Concepts as hypervolume and hypervolume indicator (e.g., see Section 3.5) should be defined in
order to make the paper self-contained. This will allow to better understand the meaning and the
values specifying the reference points for the hypervolume indicator (e.g., see Section 4.1). In
particular, without providing additional details in Section 3.5, it is very difficult for the reader to
understand the implication of the choices made in Section 4.1.
}
\answer{\TODO{tbd}}

\reviewer{
Provide some details to clarify the statements at the end of Section 4.1.
}
\answer{\TODO{tbd}}

\reviewer{
Quantifying risk equity is not clear. Are risk equity values computed according to o.f. (3), i.e., in
terms of maximum link total risk? Please clarify.
}
\answer{\TODO{tbd}}

\reviewer{
Figure 3 is too small to appreciate the depicted points. Moreover, and again, the definition of
equity risk is misleading: one wishes to maximize risk equity, while from the figure it seems that it
is preferable to have small risk equity. Of course, if in the figure ``risk equity'' is replaced with
``maximum link total risk'' the figure will make sense. The same applies to Figures 5, 6, 8, and 10.
}
\answer{\TODO{tbd}}

%%%%%%%%%%%%%%%%%%%%%%%%%%%%%%%%%%%%%%%%%%%%%%%%%%%%%%%%%%%%%%%%%%%%%%%%%
\section*{Reviewer 2}
%%%%%%%%%%%%%%%%%%%%%%%%%%%%%%%%%%%%%%%%%%%%%%%%%%%%%%%%%%%%%%%%%%%%%%%%%
\reviewer{
First of all, the reviewer believes that the paper is in
general written well.  The research idea is clearly delivered and
the method and computation results are well described.
 
The paper presents a specific implementation of EMO algorithms,
namely, HypE, for tackling the multiobjective routing problem, in
which one of the objectives is the min-max type.
 
A major deficiency of this paper is that the literature review is far
from sufficiency, either for multiobjective routing problems or for
hazardous materials transportation problems.  A quite complete
review of multiobjective shortest path problems is contained in Xie
and Waller (2012), which provides a list of multiobjective routing
papers and might give the authors some clues, for example.  An
example for hazardous materials transportation is Lindner-Dutton et
al (1991), who presents a multiobjective hazardous materials
transportation problem, which includes a risk equity objective. Here
are just two examples the reviewer randomly found. Many other papers
should be consulted and included appropriately in the literature
review.  The authors should give a more comprehensive and enhanced
discussion on the relevant literature in the revised version of the
paper.
 
Lindner-Dutton, L., Batta, R. and Karwan, M.H. (1991). "Equitable
sequencing of a given set of hazardous materials shipments."
Transportation Science, 25(2), 1991.
 
Xie, C. and Waller, S.T. (2012). Parametric search and problem
decomposition for approximating Pareto-optimal paths. Transportation
Research Part B. (In press)
}
\answer{\TODO{tbd}}
 
\reviewer{ 
Another concern is about the EMO algorithm's performance.  The
authors should give a comparative evaluation, with an exact method
as the benchmark, at least on a small example problem.  Otherwise,
readers have no idea about how much the Pareto-optimal solution set
is approximated.
}
\answer{tbd}



% 
 \bibliographystyle{plainnat}
 \footnotesize
 \bibliography{all}






\end{document}

%%
%% End of file `elsarticle-template-1a-num.tex'.
