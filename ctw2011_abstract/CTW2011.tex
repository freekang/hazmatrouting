\documentclass{ctwart}
\usepackage{amssymb}
\usepackage{color}
% \usepackage{dsfont}
\usepackage[numbers,sort&compress]{natbib}

\newcommand{\TODO}[1]{{\color{red}TODO: #1}}
\newcommand{\TODON}[1]{{\color{blue}TODO: #1}}
\newcommand{\COMMENTD}[1]{{\color{green}Dimo: #1}}
\newcommand{\COMMENTN}[1]{{\color{green}Nora: #1}}

\newcommand{\mypara}[1]{{\bfseries #1: }}

\begin{document}

\begin{frontmatter}

\title{An Evolutionary Algorithm for the Multiobjective Risk-Equity Constrained Routing Problem}


\author{Nora Touati-Moungla and Dimo Brockhoff}

\address{Laboratoire d'Informatique, \'{E}cole Polytechnique, 91128 Palaiseau Cedex, France}

\begin{keyword}
Evolutionary multiobjective optimization, Transportation of hazardous materials, Risk equity, Shortest Paths. 
\end{keyword}

\end{frontmatter}

\section{Introduction}

The transportation of hazardous materials ({\em hazmat} from now on) has received a large interest in recent years, which results from the increase in public awareness of the dangers of hazmats and the enormous amount of hazmats being transported \cite{CAR08}. The main target of this problem is to select routes from a given origin-destination pair of nodes such that the risk for the surrounding population and the environment is minimized---without producing excessive economic costs. 
%
When solving such a problem by minimizing both cost and the total risk, typically several vehicles share the same (short) routes which results in high risks associated to regions surrounding these paths whereas other regions are not affected. In this case, one may wish to distribute the risk in an equitable way over the population and the environment. Several studies consider this additional minimization of the equity risk, but most of them consist of a \emph{single} origin-destination hazmat routing for a specific hazmat, transport mode and vehicle type (see for example \cite{AKG02, CAR08}).
% 
% 
% and can be classified into the \textit{resolution-equity-based methods} and the \textit{model-equity-based methods}. In resolution-equity-based methods, the equity constraints are taken into account only indirectly via a dissimilarity index between paths which is to be maximized \cite{JOH92, LOM93, KUB97, AKG02}. Model-equity-based methods, on the other hand, formalize the risk equity as constraints directly within the model. In \cite{GOP90a, GOP90b}, for example, the authors propose an equity shortest path model that minimizes the total risk of travel while the difference between the risks imposed on any two arbitrary zones does not exceed a given threshold.
%
A more realistic \emph{multi}-commodity flow model was proposed in \citep{CAR08} where each commodity is considered as one hazmat type. The objective function is formulated as the sum of the economical cost and the cost related to the consequences of an incident for each material. To deal with risk equity, the costs are defined as functions of the flow traversing the arcs which imposes an increase of the arc's cost and risk when the number of vehicles transporting a given material increases on the arc.

The majority of all hazmat routing studies deal with a single-objective scenario although the problem itself is multiobjective in nature and it is important to study the trade-offs among the objectives. Evolutionary Multiobjective Optimization (EMO) algorithms are able to compute a set of solutions showing these trade-offs within a single algorithm run which is the reason why we propose to use them for the problem of hazmat routing in this study (Sec.~\ref{S_MEO}). Before, we formalize the routing of hazmat problem with three objectives (minimize total routing cost, total routing risk {\em and} risk equity) as a multicommodity flow model as in \cite{CAR08} since this model is the most realistic one permitting to manage several hazmat types simultaneously (Sec.~\ref{S_RECRP}).



% see for example \cite{CAR08, cgr2007, gkbk1990, za2004}, but all problem formulations include the equity risk as constraints and not as an additional objective. \citet{gian1998} on the other hand considers the equity risk as separate objective but uses a weighted sum to aggregate his four objectives into a single-objective problem. The main reason why, to the best of our knowledge, no study considered a fully multiobjective model with the equity risk as objective might be the resulting non-linearity of the problem which make classical optimization algorithms such as branch-and-cut \TODON{equity is an objective in \cite{{gian1998}} ... We solve the broblem with a branch and price algorithm because the model is a multicommodity flow model, and this model is the more adapted one for this problem because it permit to manage simultaneously many hazmat types. The min-max formulation of the objective is easily linearized. I give you the major axes of the work: (1) Solve the presented problem (optimize cost, risk and equity) (2) what are models and resolution methods proposed in the litterature (3) what is the model retained and why (in our case, the multicommodity model) (4) how to handle equity with this model ? (5) how to solve it ? (branch and price is adapted for these models, but can not hadle many objectives $->$ evolutionnary algorithms) } infeasible and led us to our choice of proposing an evolutionary multiobjective algorithm as heuristic here.

%The computation of routes with a fairly distributed risk consists in generating dissimilar origin-destination paths, i.e paths which relatively don't impact the same zones. Solution approaches 

%Section~\ref{S_RECRP} defines the problem of transporting hazardous materials with equity risk as a three-objective optimization problem with the minimization of the total routing cost, of the total routing risk {\em and} of the risk equity as the objectives---the latter broadly defined as the risk shared by a set of regions that compose the geographical area under consideration. Based on this formulation, Sec.~\ref{S_MEO} presents an evolutionary multiobjective optimization algorithm before Sec.~\ref{S_FW} concludes the abstract.


%In \cite{CAR08} was proposed a multi-commodity flow model for this problem, where each commodity is considered as one hazmat type. To deal with risk equity, the costs are defined as functions of the flow traversing the arcs, this imposes an increase of the arc's cost and risk when the number of vehicles transporting a given material increases on the arc. The major drawback of this model is that it can lead to the saturation of the arcs of shortest paths and, therefore, does not guarantee the equity target. 

%In \cite{BEL10}, the authors consider a similar model but with additional equity constraints and objectives. The objective is the minimization of the maximum risk equity imposed on the population and the environment and the model is solved using a pranch-and-price algorithm. 

%As hazmat routing is multiobjective in nature, we introduce in this work a multiobjective version of this problem. Our goal is the minimization of the total routing cost, of the total routing risk {\em and} of the risk equity, the latter broadly defined as the risk shared by a set of regions that compose the geographical area under consideration. 

%This paper is organized as follows. We describe the problem in section \ref{S_RECRP}, we give its mathematical formulation in section \ref{S_AOM}, in section \ref{S_MEO} we show how multiobjective evolutionary methods can be applied to this problem and in section \ref{S_FW} we exopse our further work.


%%%%%%%%%%%%%%%%%%%%%%%%%%%%%%%%%%%%%%%%%%%%%%%%%%%%%%%%%%%%%%%%%%%%%%%%%%%%%%%%%%%%%%%%%%%%
\section{The multiobjective risk-equity constrained routing problem}
\label{S_RECRP}

Let the transportation network be represented as a directed graph $G = (N, A)$, with $N$ being the set of nodes and $A$ the set of arcs. Let $C$ be the set of commodities, given as a set of point-to-point demands to transport a certain amount of hazmats. For any commodity $c\in C$, let $s^c$ and $t^c$ be the source node and the destination node respectively, and let $V^c$ be the number of available trucks for the transportation of commodity $c$. Each commodity is associated with a unique type of hazmat. We assume that the risk is computed on each arc of the network and is proportional to the number of trucks traversing such an arc. We consider a set $Q$ of regions, and we define $r_{ij}^{cq}$ as the risk imposed on region $q \in Q$ when the arc $(i,j) \in A$ is used by a truck for the transportation of hazmat of type $c$. We remark that we employ a notion of {\em spread risk}, in that an accidental event on arc $(i,j)$ within region $q\in Q$ may strongly affect another region $q'\in Q$. With each arc $(i,j) \in A$ a cost $c_{ij}^c$ is associated, involved by the travel of a truck of commodity $c$ on this arc.

The problem of transporting hazmat is multiobjective in nature: one usually wants to minimize the \textit{total cost} of transportation, the \textit{total risk} of transportation imposed on all regions and the {\em distributed risk}, which can be defined as a measure of risk that is shared among different regions. More specifically, for a given solution, each region $q\in Q$ will be affected by a risk $\omega_q$ which depends on the transportation patterns in all other regions. The third objective will then be $\max_{q\in Q} \omega_q$, and has to be minimized.

We introduce the integer variable $y_{ij}^c$ for the number of trucks that use arc $(i,j)$ for transporting commodity $c$. We assume a fixed number of trucks and that all trucks have the same load. The proposed model is defined as follows:
\begin{eqnarray}
   \min  & \sum_{c \in C} \sum_{(i,j) \in A} c_{ij}^{c} y_{ij}^c & \label{eq:obj1}\\
   \min  & \sum_{c \in C} \sum_{q\in Q} \sum_{(i,j) \in A} r_{ij}^{cq} y_{ij}^c & \label{eq:obj2}\\
   \min  & \max_{q\in Q} \big\{ \sum_{c \in C} \sum_{(i,j) \in A} r_{ij}^{cq} y_{ij}^c \big\} & \label{eq:obj3}\\
    s.t. & \sum_{j \in \delta^+(i)} y_{ij}^c -  \sum_{j \in \delta^-(i)} y_{ji}^c = q_i^c\quad         & \forall i \in N,  c \in  C \label{eq:flow} \\
         & y_{ij}^c \in \{0, 1, 2, \ldots\}                                                            & \forall (i,j)\in A, c\in C\label{eq:integerY}.
\end{eqnarray}

The first objective in (\ref{eq:obj1}) is a cost function and is to be minimized. The second objective, given by (\ref{eq:obj2}), minimizes the total risk on all regions and objective (\ref{eq:obj3}) minimizes the maximum risk imposed on all regions. Constraints (\ref{eq:flow}) are conservation constraints, where $\delta^-(i) = \{j \in N: (j,i)\in A\}$ and $\delta^+(i) = \{j \in N: (i,j)\in A\}$ are the direct successors and predecessors of node $i$, and $q_i^c=V^c$ if  $i=s^c$, $q_i^c=-V^c$ if  $i=t^c$ and $q_i^c=0$ otherwise.


\section{Evolutionary multiobjective optimization}
\label{S_MEO}
Evolutionary algorithms (EAs) and Evolutionary Multiobjective Optimization (EMO) algorithms in particular are general-purpose randomized search heuristics and as such well suited for problems where the objective function(s) can be highly non-linear, noisy, or even given only implicitly, e.g., by expensive simulations \citep{deb2001a,cvl2007a}. Since the third objective in the above problem formulation is nonlinear, we propose to use an EMO algorithm for the multiobjective risk-equity constrained routing problem here. Our EMO algorithm follows the standard iterative/generational cycle of an EA of mating selection, variation, objective function evaluation, and environmental selection and is build upon the state-of-the-art selection scheme in HypE \citep{bz2011a} as implemented in the PISA platform \citep{bltz2003a}. The variation operators as well as the representation of the solutions, however, have to be adapted to the problem at hand in the following way in order to fulfill the problem's constraints at all times.

\mypara{Representation}
We choose a variable length representation as it has been theoretically shown to be a good choice for multiobjective shortest paths problems \citep{horo2009a}: A solution is thereby represented by a list of paths of variable lengths with one path per truck. For the moment, we consider a fixed amount of trucks for each commodity and therefore a fixed number of paths through the network. In order to have every variable length path represent an uninterrupted path from source to destination at any time (see the constraints in (\ref{eq:flow})), we ensure all paths to always start with the source $s^c$ for the corresponding commodity $c$, ensure with the variation operator that all neighbored vertices in the path are connected by an arc, and complete each path by the shortest path between the path's actual end node and the commodity's destination node $t^c$.

\mypara{Initialization}
Initially, we start with a single solution where the paths $p$ for all trucks are empty ($p=(s^c)$). This corresponds to the situation where all trucks choose the shortest $s^c$--$t^c$ path for their assigned commodity---implying the smallest possible overall cost but a high risk along the used route(s). Nevertheless, the initial solution is already Pareto-optimal and is expected to be a good starting point for the algorithm.

\mypara{Variation}
As mutation operator, we suggest to shorten or lengthen the path of one or several trucks. In order to generate a new solution $s'$ from $s$, for each truck path, we draw a binary value $b\in\{0,1\}$ uniformly at random and create the new path $p'$ from the old one $p=(v_1=s^c,v_2,\ldots, v_l)$ as in \citep{horo2009a}:
\begin{itemize}
	\item if $b = 0$ and $l=:\textnormal{length}(p)\geq 2$, set $p' = (s^c, \ldots, v_{l-1})$	
	\item if $b = 1$ and $|V_{\textnormal{\scriptsize rem}}=\{v \in V \,|\, (v_{l}, v) \in A\}| \neq\emptyset$, choose $v_{l+1}$ from $V_{\textnormal{\scriptsize rem}}$ uniformly at random and set $p' = (v_0, \ldots, v_l, v_{l+1})$.
	\item otherwise, use the same path $p$ also in the new solution $s'$.
\end{itemize}


\section{Conclusions}
\label{S_FW}
The transportation of hazmats is an important optimization problem in the field of sustainable development and in particular the equitable distribution of risks is of high interest. Within this study, we formalize this transportation problem as the minimization of three objectives and propose to use an evolutionary algorithm to cope with the non-linear equity risk objective.

The third objective function of our problem can be rewritten by minimizing the additional variable $z$ as third objective and adding the constraints $\forall q \in Q: z \geq \sum_{c \in C} \sum_{(i,j) \in A} r_{ij}^{cq} f_{ij}^c$. Although this equivalent formulation makes the problem linear (with additional linear constraints), classical algorithms are expected to have difficulties with this formulation as well and our algorithm is supposed to be more efficient in the current formulation due to the fewer number of constraints. Note that, for the moment, the proposed EMO algorithm exists on paper only and an actual implementation has to prove in the future which additional algorithm components (such as problem-specific initialization, recombination operators, or other exact optimization (sub-)procedures) are necessary to generate solutions of sufficient quality and whether adaptively changing the number and capacity of trucks is beneficial.




% The transportation of hazmats is an important optimization problem in the field of sustainable development and in particular the equitable distribution of risks is of high interest. 

% In this study, we formalize transportation of hazmats problem as the minimization of three objectives and propose to use an evolutionary algorithm since many exact optimization procedures have difficulties with the non-linear equity risk objective. Note that our problem formulation can be rewritten by minimizing the additional variable $z$ as third objective and adding the constraints $\forall q \in Q: z \geq \sum_{c \in C} \sum_{(i,j) \in A} r_{ij}^{cq} f_{ij}^c$ \COMMENTD{this has to be adapted according to the new problem formulation}. Although this equivalent formulation makes the problem linear (with additional non-linear constraints), classical algorithms are expected to have difficulties with this formulation as well and the EMO algorithm is supposed to be more efficient in the current formulation due to the fewer number of constraints.
% 
% Note also that, for the moment, the proposed EMO algorithm only exists as a gedankenexperiment and an actual implementation has to prove in the future which additional algorithm components (such as problem-specific initialization, recombination operators, or other exact optimization (sub-)procedures) are necessary to improve the quality of the found solutions to a sufficient level.


% 
 \bibliographystyle{plainnat}
 \footnotesize
% \bibliography{hazmat}

\begin{thebibliography}{00}
\providecommand{\natexlab}[1]{#1}
\providecommand{\url}[1]{\texttt{#1}}
\expandafter\ifx\csname urlstyle\endcsname\relax
  \providecommand{\doi}[1]{doi: #1}\else
  \providecommand{\doi}{doi: \begingroup \urlstyle{rm}\Url}\fi

\bibitem[Akgun et~al.(2003)Akgun, Erkut, and Batta]{AKG02}
V. Akgun, E. Erkut and R. Batta, 
On finding dissimilar paths, 
European Journal of Operational Research 121(2):232-246, 2000.

\bibitem[Bader and Zitzler(2011)]{bz2011a}
J.~Bader and E.~Zitzler.
\newblock Hype: An algorithm for fast hypervolume-based many-objective
  optimization.
\newblock \emph{Evolutionary Computation}, 19\penalty0 (1):\penalty0 45�--76,
  2011.

\bibitem[Bleuler et~al.(2003)Bleuler, Laumanns, Thiele, and Zitzler]{bltz2003a}
S.~Bleuler, M.~Laumanns, L.~Thiele, and E.~Zitzler.
\newblock PISA---a platform and programming language independent interface for
  search algorithms.
\newblock In \emph{Evolutionary
  Multi-Criterion Optimization {(EMO~2003)}}, pages
  494--508, 2003. Springer.

\bibitem[Caramia et~al.(2008)Caramia and Dell'Omo]{CAR08}
M. Caramia and P. Dell'Olmo, 
Multiobjective management in freight logistics: increasing capacity, service level and safety with optimization algorithms,
Springer London Ltd, 2008.

\bibitem[{Coello Coello} et~al.(2007){Coello Coello}, {Lamont}, and {Van
  Veldhuizen}]{cvl2007a}
C.~A. {Coello Coello}, G.~B. {Lamont}, and D.~A. {Van Veldhuizen}.
\newblock \emph{Evolutionary Algorithms for Solving Multi-Objective Problems}.
\newblock Springer, 2007.

\bibitem[Deb(2001)]{deb2001a}
K.~Deb.
\newblock \emph{Multi-Objective Optimization Using Evolutionary Algorithms}.
\newblock Wiley, Chichester, UK, 2001.

\bibitem[Horoba(2009)]{horo2009a}
C.~Horoba.
\newblock Analysis of a simple evolutionary algorithm for the multiobjective
  shortest path problem.
\newblock In \emph{Foundations of Genetic Algorithms {(FOGA 2009)}}, pages
  113--120. ACM, 2009. 

% \bibitem{GOP90b}
% R. Gopalan, K.S. Kolluri, R. Batta and M.H. Karwan, 
% Modeling equity of risk in the transportation of hazardous materials, 
% Operations Research, 38(6):961-973, 1990.
% 
% \bibitem{KUB97}
% M. Kuby, X. Zhongyi and X. Xiaodong, 
% A minimax method for finding the $k$ best differentiated paths, 
% Geographical Analysis 29(4):298-313, 1997.

% \bibitem{LOM93}
% K. Lombard and R.L. Church, 
% The gateway shortest path problem: Generating alternative routes for a corridor location problem, 
% Geographical Systems, 1:25-45, 1993.

\end{thebibliography}

\end{document}
